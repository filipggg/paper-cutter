
\pdfoutput=1
%%%%%%%%%%%%%%%%%%%%%%%%%%%%%%%%%%%%%%%%%%%%%%%%%%%%%%%%%%%%%%%%%%%%%%%%%%%%%%%%%%%%%%%%%%%%%%%%%%%%
% [paper-cutter clarifications]
%
% DO NOT EDIT THIS FILE (unless you know what you are doing).
% THIS IS A FILE SUPPLIED BY THE TEMPLATE.
% PLEASE EDIT main.tex INSTEAD.
%
%%%%%%%%%%%%%%%%%%%%%%%%%%%%%%%%%%%%%%%%%%%%%%%%%%%%%%%%%%%%%%%%%%%%%%%%%%%%%%%%%%%%%%%%%%%%%%%%%%%%
%%
%% This is file `sample-sigconf.tex',
%% generated with the docstrip utility.
%%
%% The original source files were:
%%
%% samples.dtx  (with options: `sigconf')
%%
%% IMPORTANT NOTICE:
%%
%% For the copyright see the source file.
%%
%% Any modified versions of this file must be renamed
%% with new filenames distinct from sample-sigconf.tex.
%%
%% For distribution of the original source see the terms
%% for copying and modification in the file samples.dtx.
%%
%% This generated file may be distributed as long as the
%% original source files, as listed above, are part of the
%% same distribution. (The sources need not necessarily be
%% in the same archive or directory.)
%%
%% The first command in your LaTeX source must be the \documentclass command.
\documentclass[sigconf]{acmart}

%%
%% \BibTeX command to typeset BibTeX logo in the docs
\AtBeginDocument{%
  \providecommand\BibTeX{{%
    \normalfont B\kern-0.5em{\scshape i\kern-0.25em b}\kern-0.8em\TeX}}}

%% Rights management information.  This information is sent to you
%% when you complete the rights form.  These commands have SAMPLE
%% values in them; it is your responsibility as an author to replace
%% the commands and values with those provided to you when you
%% complete the rights form.
\setcopyright{acmcopyright}
\copyrightyear{2018}
\acmYear{2018}
\acmDOI{10.1145/1122445.1122456}

%% These commands are for a PROCEEDINGS abstract or paper.
\acmConference[Woodstock '18]{Woodstock '18: ACM Symposium on Neural
  Gaze Detection}{June 03--05, 2018}{Woodstock, NY}
\acmBooktitle{Woodstock '18: ACM Symposium on Neural Gaze Detection,
  June 03--05, 2018, Woodstock, NY}
\acmPrice{15.00}
\acmISBN{978-1-4503-XXXX-X/18/06}


%%
%% Submission ID.
%% Use this when submitting an article to a sponsored event. You'll
%% receive a unique submission ID from the organizers
%% of the event, and this ID should be used as the parameter to this command.
%%\acmSubmissionID{123-A56-BU3}

%%
%% The majority of ACM publications use numbered citations and
%% references.  The command \citestyle{authoryear} switches to the
%% "author year" style.
%%
%% If you are preparing content for an event
%% sponsored by ACM SIGGRAPH, you must use the "author year" style of
%% citations and references.
%% Uncommenting
%% the next command will enable that style.
%%\citestyle{acmauthoryear}

\newcommand\bycite[1]{in~\citet{#1}}

% In this file some settings are passed from cookiecutter to LaTeX.
% Do not edit this file, but rather change .cookiecutter.yml and
% re-apply the template.

% whether an appendix was created
\newif\ifwithappendix

\withappendixtrue

\withappendixfalse


% whether the appendix will actually be shown;
% note: in some template, the appendix cannot be rendered,
% you can use this conditional to change the content of your paper
% (for instance, to remove parts like "see Appendix")
\newif\ifappendixshown

\appendixshownfalse

\appendixshowntrue





% various custom definitions

\usepackage[T1]{fontenc}
\usepackage[utf8]{inputenc}
\usepackage[textsize=tiny]{todonotes}
\usepackage{graphicx}
\usepackage{booktabs}
\usepackage{latexsym}

% so that footnotes in tables would work
% https://tex.stackexchange.com/questions/109467/footnote-in-tabular-environment

\usepackage{footnote}
\makesavenoteenv{tabular}
\makesavenoteenv{table}
\makesavenoteenv{table*}

% for better typesetting of URLs

\usepackage{xurl}

% input without unwanted end-of-line characters

\newcommand\minput[1]{%
  \input{#1}%
  \ifhmode\ifnum\lastnodetype=11 \unskip\fi\fi}

% Gonito stuff
\usepackage{hyperref}
\usepackage{xstring}
% Format a reference to a Gonito submission
\newcommand{\gonitoref}[1]{\{\href{https://gonito.net/q/#1}{\StrMid{#1}{1}{6}}\}}
% A bare score from Gonito
\newcommand{\gonitobarescore}[1]{\minput{scores/#1.txt}}
% A score from Gonito along with a reference
\newcommand{\gonitoscore}[1]{\gonitobarescore{#1} \gonitoref{#1}}
% A reference and a score as two cells in a table
\newcommand{\gonitoentry}[1]{\gonitoref{#1} & \minput{scores/#1.txt}}



\newcommand{\code}[1]{\texttt{#1}}
\newcommand{\noqa}[1]{}
\newcommand{\noqall}[1]{}



%%% Local Variables:
%%% mode: latex

%%% TeX-master: "{{cookiecutter.paper_id}}"
%%% End:


% extra packages and definitions

\usepackage{amsmath}


%%
%% end of the preamble, start of the body of the document source.
\begin{document}

\input{metadata}

%%
%% The abstract is a short summary of the work to be presented in the
%% article.
\begin{abstract}
Sample abstract.

\end{abstract}

%%
%% The code below is generated by the tool at http://dl.acm.org/ccs.cfm.
%% Please copy and paste the code instead of the example below.
%%
\begin{CCSXML}
<ccs2012>
 <concept>
  <concept_id>10010520.10010553.10010562</concept_id>
  <concept_desc>Computer systems organization~Embedded systems</concept_desc>
  <concept_significance>500</concept_significance>
 </concept>
 <concept>
  <concept_id>10010520.10010575.10010755</concept_id>
  <concept_desc>Computer systems organization~Redundancy</concept_desc>
  <concept_significance>300</concept_significance>
 </concept>
 <concept>
  <concept_id>10010520.10010553.10010554</concept_id>
  <concept_desc>Computer systems organization~Robotics</concept_desc>
  <concept_significance>100</concept_significance>
 </concept>
 <concept>
  <concept_id>10003033.10003083.10003095</concept_id>
  <concept_desc>Networks~Network reliability</concept_desc>
  <concept_significance>100</concept_significance>
 </concept>
</ccs2012>
\end{CCSXML}

\ccsdesc[500]{Computer systems organization~Embedded systems}
\ccsdesc[300]{Computer systems organization~Redundancy}
\ccsdesc{Computer systems organization~Robotics}
\ccsdesc[100]{Networks~Network reliability}

%%
%% Keywords. The author(s) should pick words that accurately describe
%% the work being presented. Separate the keywords with commas.
\keywords{datasets, neural networks, gaze detection, text tagging}

%% A "teaser" image appears between the author and affiliation
%% information and the body of the document, and typically spans the
%% page.

% commented by F.G.
%% \begin{teaserfigure}
%%   \includegraphics[width=\textwidth]{sampleteaser}
%%   \caption{Seattle Mariners at Spring Training, 2010.}
%%   \Description{Enjoying the baseball game from the third-base
%%   seats. Ichiro Suzuki preparing to bat.}
%%   \label{fig:teaser}
%% \end{teaserfigure}

%%
%% This command processes the author and affiliation and title
%% information and builds the first part of the formatted document.
\maketitle


\section{Main}

This is a~sample paper~\cite{DBLP:journals/corr/cs-CL-0108005}.
See the experiments described \bycite{DBLP:journals/corr/cs-CL-0108005}.

Please put your content here.



\subsection{Gonito}

Gonito submission should be referenced like this
\gonitoscore{6ab4979e4629c5559feba452b7ca74c0cac89ebb}%
\footnote{Reference codes to repositories stored at
Gonito.net~\cite{gonito2016} are given in curly brackets. Such a~repository may be also accessed by going
to \url{http://gonito.net/q} and entering the code there.}


%%
%% The next two lines define the bibliography style to be used, and
%% the bibliography file.
\bibliographystyle{ACM-Reference-Format}
\bibliography{bibliography}

%%
%% If your work has an appendix, this is the place to put it.
\ifwithappendix
\appendix

% put an appendix here if needed (after bibliography)

\section{Other stuff}

This is a sample appendix.

\fi
\end{document}
\endinput
%%
%% End of file `sample-sigconf.tex'.

%%% Local Variables:
%%% mode: latex
%%% TeX-master: t
%%% End:

