
% --- Autorzy prac
% --- --- Imię i nazwisko
% --- --- Numer albumu
% --- --- Płeć (M/K)
\firstAuthor{ {{cookiecutter.main_contributor_name}} }
\firstAlbum{ {{cookiecutter.album_no}} }
\stsexFirstAuthor{M}

%\secondAuthor{Autor Drugi}
%\secondAlbum{234561}
%\stsexSecondAuthor{M}

%\thirdAuthor{Autor Trzeci}
%\thirdAlbum{345612}
%\stsexThirdAuthor{M}

%\fourthAuthor{Autor Czwarty}
%\fourthAlbum{456123}
%\stsexFourthAuthor{F}

% --- Tytuł pracy (w języku polskim i angielskim)
\titlePL{ {{ cookiecutter.paper_title }} }
\titleEN{PUT YOUR TITLE IN ENGLISH!!!!!!!!}
% --- Typ pracy (inżynierska, licencjacka, magisterska)
\type{magisterska}
% --- Wydział (wykaz skrótów):
% --- --- WA    --- Wydział Anglistyki
% --- --- WAiK  --- Wydział Antropologii i Kulturoznawstwa
% --- --- WAr   --- Wydział Archeologii
% --- --- WB    --- Wydział Biologii
% --- --- WCh   --- Wydział Chemii
% --- --- WFPiK --- Wydział Filologii Polskiej i Klasycznej
% --- --- WFi   --- Wydział Filozoficzny
% --- --- WF    --- Wydział Fizyki
% --- --- WGSE  --- Wydział Geografii Społeczno-Ekonomicznej i Gospodarki Przestrzennej
% --- --- WH    --- Wydział Historii
% --- --- WMiI  --- Wydział Matematyki i Informatyki
% --- --- WNGiG --- Wydział Nauk Geograficznych i Geologicznych
% --- --- WNoS  --- Wydział Nauk o Sztuce
% --- --- WNPiD --- Wydział Nauk Politycznych i Dziennikarstwa
% --- --- WN    --- Wydział Neofilologii
% --- --- WPiK  --- Wydział Psychologii i Kognitywistyki
% --- --- WPAK  --- Wydział Pedagogiczno-Artystyczny w Kaliszu
% --- --- WPiA  --- Wydział Prawa i Administracji
% --- --- WS    --- Wydział Socjologii
% --- --- WSE   --- Wydział Studiów Edukacyjnych
% --- --- WT    --- Wydział Teologiczny
% --- --- IKE   --- Instytut Kultury Europejskiej w Gnieźnie
\faculty{WMiI}
% --- Kierunek (w mianowniku)
\field{ {{ cookiecutter.discipline }} }
% --- Specjalność (w formie mianownikowej)
% --- (ustaw puste, jeśli bez specjalności)
\specialty{ {{ cookiecutter.specialization }} }
% --- Promotor (w dopełniaczu)
\supervisor{ {{ cookiecutter.supervisor }} }
% --- Data złożenia pracy (Miasto, miesiąc rok)
\newcommand{\monthname}{\ifcase \month \or styczeń\or luty\or marzec\or kwiecień\or maj%
\or czerwiec\or lipiec\or sierpień\or wrzesień\or październik\or listopad\or grudzień\fi}

\date{Poznań, \monthname{} \the\year}
% --- Zgoda na udostępnienie pracy w czytelni (TAK/NIE)
\stread{TAK}
% --- Zgoda na udostępnienie pracy w zakresie ochrony (TAK/NIE)
\stprotect{TAK}
% --- Data podpisania oświadczenia (Miasto, data)
\stdate{Poznań, \today{} r.}
