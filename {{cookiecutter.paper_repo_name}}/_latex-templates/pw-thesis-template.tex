
\pdfoutput=1
\documentclass[a4paper,11pt,twoside]{report}
% THIS FILE SHOULD BE COMPILED BY pdfLaTeX

% ----------------------   PREAMBLE PART ------------------------------

% In this file some settings are passed from cookiecutter to LaTeX.
% Do not edit this file, but rather change .cookiecutter.yml and
% re-apply the template.

% whether an appendix was created
\newif\ifwithappendix

\withappendixtrue

\withappendixfalse


% whether the appendix will actually be shown;
% note: in some template, the appendix cannot be rendered,
% you can use this conditional to change the content of your paper
% (for instance, to remove parts like "see Appendix")
\newif\ifappendixshown

\appendixshownfalse

\appendixshowntrue



% extra packages and definitions

\usepackage{amsmath}




% various custom definitions

\usepackage[T1]{fontenc}
\usepackage[utf8]{inputenc}
\usepackage[textsize=tiny]{todonotes}
\usepackage{graphicx}
\usepackage{booktabs}
\usepackage{latexsym}

% so that footnotes in tables would work
% https://tex.stackexchange.com/questions/109467/footnote-in-tabular-environment

\usepackage{footnote}
\makesavenoteenv{tabular}
\makesavenoteenv{table}
\makesavenoteenv{table*}

% for better typesetting of URLs

\usepackage{xurl}

% input without unwanted end-of-line characters

\newcommand\minput[1]{%
  \input{#1}%
  \ifhmode\ifnum\lastnodetype=11 \unskip\fi\fi}

% Gonito stuff
\usepackage{hyperref}
\usepackage{xstring}
% Format a reference to a Gonito submission
\newcommand{\gonitoref}[1]{\{\href{https://gonito.net/q/#1}{\StrMid{#1}{1}{6}}\}}
% A bare score from Gonito
\newcommand{\gonitobarescore}[1]{\minput{scores/#1.txt}}
% A score from Gonito along with a reference
\newcommand{\gonitoscore}[1]{\gonitobarescore{#1} \gonitoref{#1}}
% A reference and a score as two cells in a table
\newcommand{\gonitoentry}[1]{\gonitoref{#1} & \minput{scores/#1.txt}}



\newcommand{\code}[1]{\texttt{#1}}
\newcommand{\noqa}[1]{}
\newcommand{\noqall}[1]{}



%%% Local Variables:
%%% mode: latex

%%% TeX-master: "{{cookiecutter.paper_id}}"
%%% End:


% ------------------------ ENCODING & LANGUAGES ----------------------

\usepackage[utf8]{inputenc}
%\usepackage[MeX]{polski} % Not needed unless You have a name with polish symbols or sth
\usepackage[T1]{fontenc}
\usepackage[english, polish]{babel}


\usepackage{amsmath, amsfonts, amsthm, latexsym} % MOSTLY MATHEMATICAL SYMBOLS
\usepackage{ulem} %cross-outs
\usepackage{tabularx} %tables with width setting
\usepackage{hyperref} %hyperlinks, for table of contents

%copy-paste from stack overflow to use vertical labels in tables
\usepackage{array,graphicx}
\usepackage{booktabs}
\usepackage{pifont}
\usepackage{float}

\newcommand*\rot{\rotatebox{90}}
\newcommand*\OK{\ding{51}}

\usepackage[final]{pdfpages} % INPUTING TITLE PDF PAGE - GENERATE IT FIRST!
%\usepackage[backend=bibtex, style=verbose-trad2]{biblatex}

% ---------------- MARGINS, INDENTATION, LINESPREAD ------------------

\usepackage[inner=20mm, outer=20mm, bindingoffset=10mm, top=25mm, bottom=25mm]{geometry} % MARGINS

\usepackage[none]{hyphenat}
\hyphenpenalty=750 % kara za łamanie słów
\hyphenpenalty 10000
\exhyphenpenalty 10000
\hyphenation{state-of-the-art}



\linespread{1.5}
\allowdisplaybreaks         % ALLOWS BREAKING PAGE IN MATH MODE

\usepackage{indentfirst}    % IT MAKES THE FIRST PARAGRAPH INDENTED; NOT NEEDED
\setlength{\parindent}{5mm} % WIDTH OF AN INDENTATION


%---------------- RUNNING HEAD - CHAPTER NAMES, PAGE NUMBERS ETC. -------------------

\usepackage{fancyhdr}
\pagestyle{fancy}
\fancyhf{}
% PAGINATION: LEFT ALIGNMENT ON EVEN PAGES, RIGHT ALIGNMENT ON ODD PAGES
\fancyfoot[LE,RO]{\thepage}
% RIGHT HEADER: zawartość \rightmark do lewego, wewnętrznego (marginesu)
\fancyhead[LO]{\sc \nouppercase{\rightmark}}
% lewa pagina: zawartość \leftmark do prawego, wewnętrznego (marginesu)
\fancyhead[RE]{\sc \leftmark}

\renewcommand{\chaptermark}[1]{\markboth{\thechapter.\ #1}{}}

% HEAD RULE - IT'S A LINE WHICH SEPARATES HEADER AND FOOTER FROM CONTENT
\renewcommand{\headrulewidth}{0 pt} % 0 MEANS NO RULE, 0.5 MEANS FINE RULE, THE BIGGER VALUE THE THICKER RULE


\fancypagestyle{plain}{
  \fancyhf{}
  \fancyfoot[LE,RO]{\thepage}

  \renewcommand{\headrulewidth}{0pt}
  \renewcommand{\footrulewidth}{0.0pt}
}



% --------------------------- CHAPTER HEADERS ---------------------

\usepackage{titlesec}
\titleformat{\chapter}
  {\normalfont\Large \bfseries}
  {\thechapter.}{1ex}{\Large}

\titleformat{\section}
  {\normalfont\large\bfseries}
  {\thesection.}{1ex}{}
\titlespacing{\section}{0pt}{30pt}{20pt}


\titleformat{\subsection}
  {\normalfont \bfseries}
  {\thesubsection.}{1ex}{}


% ----------------------- TABLE OF CONTENTS SETUP ---------------------------

\def\cleardoublepage{\clearpage\if@twoside
\ifodd\c@page\else\hbox{}\thispagestyle{empty}\newpage
\if@twocolumn\hbox{}\newpage\fi\fi\fi}


% THIS MAKES DOTS IN TOC FOR CHAPTERS
\usepackage{etoolbox}
\makeatletter
\patchcmd{\l@chapter}
  {\hfil}
  {\leaders\hbox{\normalfont$\m@th\mkern \@dotsep mu\hbox{.}\mkern \@dotsep mu$}\hfill}
  {}{}
\makeatother

\usepackage{titletoc}
\makeatletter
\titlecontents{chapter}% <section-type>
  [0pt]% <left>
  {}% <above-code>
  {\bfseries \thecontentslabel.\quad}% <numbered-entry-format>
  {\bfseries}% <numberless-entry-format>
  {\bfseries\leaders\hbox{\normalfont$\m@th\mkern \@dotsep mu\hbox{.}\mkern \@dotsep mu$}\hfill\contentspage}% <filler-page-format>

\titlecontents{section}
  [1em]
  {}
  {\thecontentslabel.\quad}
  {}
  {\leaders\hbox{\normalfont$\m@th\mkern \@dotsep mu\hbox{.}\mkern \@dotsep mu$}\hfill\contentspage}

\titlecontents{subsection}
  [2em]
  {}
  {\thecontentslabel.\quad}
  {}
  {\leaders\hbox{\normalfont$\m@th\mkern \@dotsep mu\hbox{.}\mkern \@dotsep mu$}\hfill\contentspage}
\makeatother



% ---------------------- TABLES AD FIGURES NUMBERING ----------------------

\renewcommand*{\thetable}{\arabic{chapter}.\arabic{table}}
\renewcommand*{\thefigure}{\arabic{chapter}.\arabic{figure}}


% ------------- DEFINING ENVIRONMENTS FOR THEOREMS, DEFINITIONS ETC. ---------------

\makeatletter
\newtheoremstyle{definition}
{3ex}%                           % Space above
{3ex}%                           % Space below
{\upshape}%                      % Body font
{}%                              % Indent amount
{\bfseries}%                     % Theorem head font
{.}%                             % Punctuation after theorem head
{.5em}%                          % Space after theorem head, ' ', or \newline
{\thmname{#1}\thmnumber{ #2}\thmnote{ (#3)}}
\makeatother

\theoremstyle{definition}
\newtheorem{theorem}{Theorem}[chapter]
\newtheorem{lemma}[theorem]{Lemma}
\newtheorem{example}[theorem]{Example}
\newtheorem{proposition}[theorem]{Proposition}
\newtheorem{corollary}[theorem]{Corollary}
\newtheorem{definition}[theorem]{Definition}
\newtheorem{remark}[theorem]{Remark}

% --------------------- END OF PREAMBLE PART (MOSTLY) --------------------------





% -------------------------- USER SETTINGS ---------------------------

\input{metadata}

\begin{document}
\sloppy
\selectlanguage{english}

\includepdf[pages=-]{titlepage} % THIS INPUTS THE TITLE PAGE


% ------------------ PAGE WITH SIGNATURES --------------------------------

\thispagestyle{empty}\newpage
\null

\vfill

\begin{center}
\begin{tabular}[t]{ccc}
............................................. & \hspace*{100pt} & .............................................\\
supervisor's signature & \hspace*{100pt} & author's signature
\end{tabular}
\end{center}



% ---------------------------- ABSTRACTS -----------------------------

{
\begin{abstract}

\begin{center}
\title
\end{center}

Sample abstract.

\\

\noindent \textbf{Keywords:} some keyword, another keyword

\end{abstract}
}

\null\thispagestyle{empty}\newpage


{\selectlanguage{polish}
\begin{abstract}

\begin{center}
\tytul
\end{center}

Streszczenie po polsku (dla niektórych szablonów).

\\

\noindent \textbf{Słowa kluczowe:} jedno słowo kluczowe, inne słowo kluczowe

\end{abstract}
}


% --------------------------- DECLARATION ------------------------------------


\null\thispagestyle{empty}\newpage

\null \hfill Warsaw, ..................\\

\par\vspace{5cm}

\begin{center}
Declaration
\end{center}

I hereby declare that the thesis entitled ,,\title '', submitted for the \type ~degree, supervised  by \supervisor , is entirely my original work apart from the recognized reference.
\vspace{2cm}

\begin{flushright}
  \begin{minipage}{50mm}
    \begin{center}
      ..............................................

    \end{center}
  \end{minipage}
\end{flushright}

\thispagestyle{empty}
\newpage

\null\thispagestyle{empty}\newpage
% ------------------- 4. Spis treści ---------------------
% \selectlanguage{english} - for English
\pagenumbering{gobble}
\tableofcontents
\thispagestyle{empty}
\newpage % IF YOU HAVE EVEN QUANTITY OD PAGES OF TOC, THEN REMOVE IT OR ADD \null\newpage FOR DOUBLE BLANK PAGE BEFORE INTRODUCTION


% -------------------- THE BODY OF THE THESIS --------------------------------

\null\thispagestyle{empty}\newpage
\pagestyle{fancy}
\pagenumbering{arabic}
\setcounter{page}{11}


\section{Main}

This is a~sample paper~\cite{DBLP:journals/corr/cs-CL-0108005}.
See the experiments described \bycite{DBLP:journals/corr/cs-CL-0108005}.

Please put your content here.



\subsection{Gonito}

Gonito submission should be referenced like this
\gonitoscore{6ab4979e4629c5559feba452b7ca74c0cac89ebb}%
\footnote{Reference codes to repositories stored at
Gonito.net~\cite{gonito2016} are given in curly brackets. Such a~repository may be also accessed by going
to \url{http://gonito.net/q} and entering the code there.}



% ------------------------------- BIBLIOGRAPHY ---------------------------
% LEXICOGRAPHICAL ORDER BY AUTHORS' LAST NAMES
% FOR AMBITIOUS ONES - USE BIBTEX
\bibliographystyle{unsrt}
\bibliography{bibliography}



\pagenumbering{gobble}
\thispagestyle{empty}



% ----------------------- LIST OF SYMBOLS AND ABBREVIATIONS ------------------
\chapter*{List of symbols and abbreviations}

\begin{tabular}{cl}
? & ???
\end{tabular}
\\
\thispagestyle{empty}


% ----------------------------  LIST OF FIGURES --------------------------------
\listoffigures
\thispagestyle{empty}
If you don't need it, delete it.


% -----------------------------  LIST OF TABLES --------------------------------
\renewcommand{\listtablename}{Spis tabel}
\listoftables
\thispagestyle{empty}
If you don't need it, delete it.

% -----------------------------  LIST OF APPENDICES ---------------------------
\chapter*{List of appendices}
\begin{enumerate}
\item Appendix 1
\item Appendix 2
\item In case of no appendices, delete this part.
\end{enumerate}
\thispagestyle{empty}


\end{document}

%%% Local Variables:
%%% mode: latex
%%% TeX-master: t
%%% End:

